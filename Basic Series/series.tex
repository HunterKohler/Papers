\documentclass[11pt]{article}
\usepackage{amsmath}
\usepackage{amsfonts} 
\usepackage{enumitem}
\usepackage{mathtools}
\usepackage{refcount}
\usepackage[bottom]{footmisc}

\title{Notes on Introductory Series}
\author{John Hunter Kohler}
\date{March 2020}


\newcommand{\define}{\coloneqq}
\newcommand{\abs}[1]{\left| #1 \right|}
\newcommand{\inreal}{\in\mathbb{R}}
\newcommand{\innat}{\in\mathbb{N}}

\begin{document}
\maketitle

\quad Though the formal definition of sequences and series limits will not be given. The notation used will be standard and 
informal. No proofs will be given similar to the entirety of this document. Assume all sequences and series are real.

\section*{Foundation of Sequences}
	This secition focuses on the basics of sequences and methods of computation on the infinite terms of some sequence. 
	A sequence $\{a_n\}$ is said to be convergent if
		\begin{equation}
			\lim_{n\to\infty} a_n \inreal
		\end{equation}
	If the above property is not held, the sequence is said to be divergent. When computing limits of two 
	convergent sequences $\{a_n\},\{b_n\}$, it is possible to seperatly evaluate the sequences.
		\begin{equation}
			\lim_{n\to\infty} f(a_n,b_n) = f(\lim_{n\to\infty} a_n, \lim_{n\to\infty} b_n)
		\end{equation}
	A simple tool for the evaluation of limits upon convergent series is the squeeze theorem for series.
		\begin{gather}
			a_n <= c_n <= b_n\,,\ \text{for} \ n > (N\innat) \\ 
			\lim_{n\to\infty} a_n = \lim_{n\to\infty} b_n = L
			\implies \lim_{n\to\infty} c_n = L
		\end{gather}
	Sequences, like functions, are said to be monotonic if they are strictly increasing or strictly decreasing.
	Like functions, sequences may be bounded above or below by some constant. If a sequence is bounded and monotonic, 
	then it must be convergent\footnote{Obviously, the same can be said about those sequences which are simply strictly increasing and 
	bounded above as well as strictly decreasing and bounded below.}.

\section*{Convergence of Series}
	Here, let us take a look at simple rules for figuring if a series converges or diverges. Obviously a convergent series is the sum of
	every term of a series such that the value is finite. If some series does not exist or is infinite, it is said to be divergent.
	A simple proposition is that
		\begin{equation}
			\text{If} \ \sum a_n \ \text{converges, then} \ \lim_{n\to\infty} a_n = 0.
		\end{equation}

	\subsection*{Absolute and Conditional Convergence}
		Now recognize two more properties of series. If some series $\sum a_n$ is convergent yet $\sum |a_n|$ is not, then the series $\sum a_n$
		is said to be conditionally convergent. If, along with $\sum |a_n|$, $\sum a_n$ is convergent, we say that the series is absolutly
		convergent. Otherwise, the series is divergent. From this it is important to recognize one thing: if a series is conditionally convergent,
		then when algabraically rearranging\footnote{For the purposes of computation.} its explicit terms will allow you to, 
		without logical error, compute the sum of the series as any real number.

	\subsection*{Geometric Sums and P-Series}
		Firstly define a partial sum -this usually uses confusing notation- as
			\begin{equation}
				S_n \define \sum_{i = 0}^{n}a_i
			\end{equation}
		The lower bound of the sum may change based on context, but this is the general notation with the exception of the differing use of capitalization
		of $s$. For a geometric sum, based on a sequence where the ratio between each term is a constant, the partial sum formula is slightly unintuitive,
		though important. It is vital in the proof of the infinite sum, though that will not be exemplified.
			\begin{gather}
				\text{for $a,r$ constant} \ S_n= \sum_{i = 0}^{n} ar^i = \dfrac{a(1-r^n)}{1-r} \\
				\lim_{n\to\infty} S_n = \dfrac{a}{1-r} \ \text{when} \ |r| < 1 
			\end{gather}
		As you will see later, this is a form of a power series, with radius of convergence\footnote{Put simply, as expressed above, the infinite
		geometric sum only converges for $|r| < 1$.} one. The other series to look at of the form for which we know convergence is the p-series
		\footnote{The special case is $p = 1$ which is the harmonic series. Not discussed here.}. 
			\begin{equation}
				\sum_{n=1}^{\infty} \frac{1}{n^p} \ \text{converges for}\ p > 1
			\end{equation}

	\subsection*{Integral Test}
		If there is some continuous, positive, strictly decreasing function $f(x)$ on,
			\begin{align}
				x\in[k,&\infty)\ \text{and}\ f(n) = a_n,\, n\innat,\ \text{then}\\
				&\int_{k}^{\infty} f(x)dx\ \text{and}\ \sum_{n = k}^{\infty} a_n\ \text{converge or diverge together.}
			\end{align}

	\subsection*{Comparison Test}
		This test is similar to, and essentially a lemma\footnote{Due to the fact a integral would simply be the 
		application of this theorem through its definition of a Reimann Sum.} of, the last theorem: 
			\begin{align}
				\text{If there is some}\ \sum a_n ,\, &\sum b_n \ \text{for} \ a_n \geq b_n \geq 0,\, a_n \leq b_n,\,n\innat, \\
				\text{then} \ &\sum b_n \ \text{converges} \implies \sum a_n \ \text{converges,} \\
				\text{and} \ &\sum a_n \ \text{diverges} \implies \sum b_n \ \text{diverges.}
			\end{align}

	\subsection*{Limit Comparison Test}
		By taking the associated sequences of two series and compair there infinite terms, a similar association of converge as before can
		be formed\footnote{These associations will let us extend knowledge of convergence from what we have proved to unknowns more easily.
		Also this would give the option of working the more simple of the two series.}. 
		\begingroup
			\begin{align}
				\text{Let} \sum a_n,\,\sum b_n \ \text{such that}\ &a_n \geq 0,\, b_n > 0, \\
				n\innat,\ &\text{and let} \ C \define \lim_{n\to\infty} \dfrac{a_n}{b_n}\\
				\text{If} \ C \inreal^{+} \ \text{then} \ \sum a_n \ \text{and}\ &\sum b_n \\
				\text{conver}&\text{ge }\text{or diverge together.}
			\end{align}
		\endgroup

	\subsection*{Alternating Series Test}
		To deal with series of the form below, this test is created. Once used, if it is possible to do so, another test may be required
		to determine the properties of the subseries. An alternating series is of the form:
			\begin{equation}
				a_n=(-1)^{n}b_n \ \text{or} \ a_n=(-1)^{n+1}b_n, \ \text{for} \ (b_n \geq 0 \ \text{or} \ b_n \leq 0)\footnotemark{\label{note1}},\,n\innat
			\end{equation}
			\footnotetext{\label{note1}For ($b_n \geq 0$ and strictly decreasing), or ($b_n \leq 0$ and strictly increasing). 
			Note that $\sum b_n$ does not have to converge.}
		Furthermore, the test states:
			\begin{equation}
				\text{If} \lim_{n\to\infty} b_n = 0 \ \text{and} \{b_n\} \ \text{is monotonic\footnote[\getrefnumber{note1}],} \ 
				\sum_{n=1}^{\infty} a_n \ \text{is convergent.}
			\end{equation}

	\subsection*{Ratio Test}
		The ratio test will be the best tool presented here to deal with sequences involving factorials as well as rational equations of polynomials.
		By reviewing the test below you will see that it takes a form in which, for the described cases, we may cancel out confounding terms.
			\begin{align}
				\text{Let} \ L \define &\lim_{n\to\infty} \abs{\dfrac{a_{n+1}}{a_n}} \\
				\text{If} \ L < 1, &\sum a_n \ \text{is absolutly convergent.} \\
				\text{If} \ L > 1, &\sum a_m \ \text{is divergent.} \\
				\text{If} \ L = 1, &\text{ the test is inconclusive.}
			\end{align}

	\subsection*{Root Test}
		This test is similar to the ratio test. Almost\footnote{Most cases that do not abide by this principle should not be presented
		in an introductory course. It may be worthwile checking both either way as either one may have been applied incorrectly} 
		all sequences produce the same results in the ratio and root test. This test is best utilized for series similar to
		$a_n = (b_n)^n$. The test functions thus:
			\begin{align}
				\text{Let} \ L \define &\lim_{n\to\infty} \abs{a_n}^{\frac{1}{n}} \\
				\text{If} \ L < 1, &\sum a_n \ \text{is absolutly convergent.} \\
				\text{If} \ L > 1, &\sum a_n \ \text{is divergent.} \\
				\text{If} \ L = 1, &\text{ the test is inconclusive.}
			\end{align}

\section*{Series Estimation}
	\subsection*{Bounding by Integration}
		Utilizing function integrals, we can find bounds for the value of an infinite series. Even if the representation of a series can not
		be integrated, we know enough integral approximation tequniques to approximate the improper integrals. Using notation of the 
		partial sum $S_n$, we may manually compute the partial sum of a series up to some point. Next, define the $E_n$ as below as what
		we call the remaining sum -for obvious reasons-.
			\begin{equation}
				E_n \define S - S_n = \sum_{i = 0}^{\infty} a_i - \sum_{i = 0}^{n} a_i = \sum_{i = n + 1}^{\infty} a_i
			\end{equation}
		Now it is possible to bound $E_n$ by way of integration for some function $f(x)$ that satisfies all the requirments of the Integral Test.
		To complete the series bound, simply recognize the following:
			\begin{align}
				\int_{n + 1}^{\infty} f(x) \, dx \leq E_n = S - S_n \leq \int_{n}^{\infty} f(x) \, dx\\
				S_n + \int_{n + 1}^{\infty} f(x) \, dx \leq S \leq S_n + \int_{n}^{\infty} f(x) \, dx
			\end{align}
		One more extension can be made to this method; namely that if we utilize a series $b_n$ that functions, in relation to $a_n$, similarly
		to our convergence test by comparison. This can make the computation of the approximation easier by trading in accuracy.

	\subsection*{Approximating Alternating Series}
		The equation given below is derived from the proof of the Alternating Series Test. It should be intuitive however due to, again mirroring
		the conditions of the test, $b_n$ must be strictly decreasing and positive, or the inverse. Understand that $b_n$ is our subseries of
		the alternating series. The sum and partial sums below are of the seres of interest $a_n$.
			\begin{align}
				\abs{E_n} = \abs{S - S_n} \leq b_{n + 1}
			\end{align}
		It is important to realize that $E_n$ is simply the error of our partial sum. From above we know that we can bound the error and therefor
		simply bound the total sum of the sequence:
			\begin{align}
				S_n - \abs{E_n} \leq S \leq S_n + \abs{E_n}
			\end{align}

	\subsection*{Bounding by Ratio}
		As goes all approximations in this section, we must have some $a_n$ that follows the requirments of the corresponding test -here, the 
		ratio test-. The formulas below are notably useful for bounding $E_n$ for sequences which are best analyzed with the Ratio Test. A last
		obvious precondition is that $a_n$ converges.
			\begin{align}
				\text{Define a new geometric sequ} &\text{ence by:} \ \ r_n \define \dfrac{a_{n+1}}{a_n} \\
					\text{Case $1$: Sequence}& \ {r_n} \ \text{is decreasing.} \\
						\abs{E_n}& \leq \dfrac{a_{n + 1}}{1 - r_{n + 1}} \\
					\text{Case $2$: Sequence}& \ {r_n} \ \text{is increasing.} \\
						\abs{E_n}& \leq \dfrac{a_{n + 1}}{1 - L}
			\end{align}
		These are the only two cases that happen due to $\{r_n\}$ being convergent if its ratio\footnote{\label{ratioNote}When speaking about the ratio 
		of $\{r_n\}.$, it is meant that for some $n$, namely that for which we choose $E_n$, has such a ratio} $r_n < 1$. In case two, 
		though the geometric series $\{r_n\}$ has a ratio greater than 1, we know that as $n$ goes to infinity, that this become less that one, so we
		use $L$ as the general ratio\footnotemark[\getrefnumber{ratioNote}].

\section*{Power Series}
	The general\footnote{This is the general power series along $\mathbb{R}$. Things such as radius of convergence make more sense when extended
	to the complex generalization; for example, this radius truly refers to the disk in $\mathbb{C}$ for which some series representation of $p(z)$ 
	converges} power series is thus for $a$ constant and sequence $\{c_n\}$:
		\begin{equation}
			p(x) = \sum_{n = 0}^{\infty} c_n (x - a)^n
		\end{equation}
	Let us call $\{c_n\}$ the sequence of coeffecients of $p(x)$. Breifly mentioned in the section on geometric series, 
	the radius of convergence $R$ is the bound\footnote{Meaning that $\abs{x} \leq R$ 
	or $\abs{x} < R$.} on $x$ for which $p(x)$ converges. Furthermore, we may call the range of $x$ for which $p(x)$ converges, its interval
	of convergence. When attempting to find the radius of convergence for some series, the end goal tends to be to find some inequality
	involving $x$ in solitude. To do this one would use one or multiple of the tests described in the section of series convergence. The ratio
	test is a particularly effective tool in this case. It is not always the case that $\abs{x} = R$ is not in the interval of convergence, thus
	one must find both $R$ and check the endpoints of said interval. Note that $R$ is in $\overline{\mathbb{R}}$, though $x$ only varies 
	along real numbers.This implies two important points. The first of which is $x$ will have an interval of all reals for $R=\infty$. Next, 
	even in the case of $R = 0$, the interval of $x$ will always be non null due to $p(a) = 0$.

	\subsection*{Expressing Analytic Functions}
		We will not describe the definition of an analytic function beyond saying that they are functions that can be expressed locally as
		a power series. Below\footnote{All  will be easier to prove and recognize after learning the Taylor series. Note that 
		the series below are Maclauren series.} are some frequent analytic functions\footnote{In non general form for the purposes of simplicity.} 
		described by their power series.
			\begin{align}
				(1 + x)^k &= \sum^{n = 0}_{\infty} \binom{k}{n} x^n = 1 + kx + \dfrac{k(k-1)}{2!} + \cdots
					& \abs{x} < 1\\
				e^x &= \sum_{n = 0}^{\infty} \dfrac{x^n}{n!} = 1 + x + \dfrac{x^2}{2!} + \dfrac{x^3}{3!} + \dfrac{x^4}{4!} 
				+ \dfrac{x^5}{5!} + \dfrac{x^6}{6!}\cdots 
					&  x\inreal\\
				sin(x) &= \sum_{n = 0}^{\infty} (-1)^n \dfrac{x^{2n+1}}{(2n+1)!} = x - \dfrac{x^3}{3!} + \dfrac{x^5}{5!} - \dfrac{x^7}{7!} + \cdots
					& x\inreal \\
				cos(x) &= \sum_{n = 0}^{\infty} (-1)^n \dfrac{x^{2n}}{(2n)!} = 1 - \dfrac{x^2}{2!} + \dfrac{x^4}{4!} - \dfrac{x^6}{6!} 
				+ \dfrac{x^8}{8!} - \cdots
					& x\inreal \\
				sinh(x) &= \sum_{n = 0}^{\infty} (-1)^n \dfrac{x^{2n+1}}{(2n+1)!} = x + \dfrac{x^3}{3!} + \dfrac{x^5}{5!} + \dfrac{x^7}{7!} + \cdots
					& x\inreal \\
				cosh(x) &= \sum_{n = 0}^{\infty} (-1)^n \dfrac{x^{2n}}{(2n)!} = 1 + \dfrac{x^2}{2!} + \dfrac{x^4}{4!} + \dfrac{x^6}{6!} 
				+ \dfrac{x^8}{8!} + \cdots
					& x\inreal
			\end{align}
		The next step is to bring calculus into our representations. Obviously the integral or derivative of a series is simply that of
		each individual term. Our first theorem is thus:
			\begin{gather}
				\text{For some differentiable, continuous} \ f(x) = \sum_{n = 0}^{\infty} c_n (x - a)^n \\
				\dfrac{df}{dx} \ \text{and} \ \int f(x) \, dx \ \text{have the equivalent $R$ to} \ f(x)
			\end{gather}
		Recognize in many situations it may be benificial to the take the $n$th derivative of a function so to find a easily recognizable
		power series, then integrate $n$ times -while solving for the integration constants- to find the power series of the original function.
		A simple example is the power series of $ln(1-x)$. In the next section we will learn how to find the power series of this easily, but we 
		may also find it through simply differentiating it so that it is represented as the geometric series.

	\subsection*{Taylor Series}
		Put simply, the Taylor series generalizes a power series to any infinitly differentiable function. These series still have radii of convergence
		and are said to be centered around some $a$\footnote{This central value -this again makes more sense over a complex feild - will also be notated 
		frequently as $x_0$. See the below that $a$ is simply some value on the domain of $f$ for which we will find $f(a)$.}. Assume $f(x)$ is a 
		infinitly differentiable function that is continuous and differentiable at $a$.
			\begin{align}
				f(x) = \sum_{i = 0}^{\infty} \dfrac{f^{(i)}}{i!} (x - a)^i 
					 = f(a) + f^{'}(a)(x - a) + \cdots
			\end{align}
		Note that the taylor is simply the semi-general power series described with $\{c_n\} = \{f^{(n)}(a)\}$.\footnote{To generalize this series we 
		obviously must find a general form for $f^{(n)}$, which is not always the easiest task} As described before, this equation
		only holds true for $x\in R_a$, where we say $R_a$ is the radius of convergence around $a$. If we have some series, we can approximate
		the value of $f(x),\, x\in R_a$ to any level of specificity. This is extremely usefull in the cases of trancendental functions that can
		not be calculated. I will again mention that a Taylor series around $a = 0$ is reffered to as a Maclauren series. To finish, one peice
		of frequent of notation is that of the $n$th degree of a Taylor series denoted by $T_n$.
			\begin{equation}
				T_n = \sum_{i = 0}^{n} \dfrac{f^{(i)}}{i!} (x - a)^i 
					 = f(a) + f^{'}(a)(x - a) + \cdots + \dfrac{f^{(n)}}{n!}(a)(x - a)^n
			\end{equation}
\end{document}















